% Search for all the places that say "PUT SOMETHING HERE".

\documentclass[11pt]{article}
\usepackage{amsmath,textcomp,amssymb,geometry,graphicx,enumerate}

\def\Name{Quoc Thai Nguyen Truong}  % Your name
\def\SID{24547327}  % Your student ID number
\def\Login{cs161-di} % Your login (your class account, cs170-xy)
\def\Homework{1}%Number of Homework
\def\Session{Spring 2015}


\title{CS161--Spring 2015 --- Solutions to Homework \Homework}
\author{\Name, SID \SID, \texttt{\Login}}
\markboth{CS161--\Session\  Homework \Homework\ \Name}{CS161--\Session\ Homework \Homework\ \Name, \texttt{\Login}}
\pagestyle{myheadings}

\newenvironment{qparts}{\begin{enumerate}[{(}a{)}]}{\end{enumerate}}
\def\endproofmark{$\Box$}
\newenvironment{proof}{\par{\bf Proof}:}{\endproofmark\smallskip}

\textheight=9in
\textwidth=6.5in
\topmargin=-.75in
\oddsidemargin=0.25in
\evensidemargin=0.25in

\newcommand{\tab}{\hspace*{2em}}


\begin{document}
\maketitle

Collaborators: God

\section*{Problem 1}

\begin{qparts}
\item
False\\
In order to keep u secure, you have to keep your code and system secret.
\item
False\\
format string bugs can be xploited for control-hijacking attacks by hackers.

\item
False\\
I'm not sure about this. So I prayed, and "False" is my final answer :)

\item
False\\
we can't prevents every attack, but not all of them. 
\item
False\\
calling system() is a very bad way to run because it runs the default shell and passes your program as arguments. 

\item
False\\
you can't guarantee a perfect program without a bug.



\end{qparts}

\newpage
\section*{Problem 2}
\begin{qparts}
\item
\boxed{Buggy} at line 5.\\
At line 5, each memmove copies string of size(p) to p+2, so it can overflow 1 byte. Therefore, the number of bytes overflow is the number of "\textbackslash n" (newline character). Hence, attacker can inject shell-code into the buffer overflow and overwrite the return address, which can jump and execute the shell-code.

\item
\boxed{Not\ Buggy}\\


\item
\boxed{Buggy} at line 12.\\
\\
This is buffer overrun. If track = curcd$\rightarrow$numtracks, this can write past curcd$\rightarrow$tracklen array.\\
If track = curcd$\rightarrow$numtracks, then I can make newtracklen be the address of same shell-code that lives somewhere in the memory.Therefore, we can see that this will overwrite curcd$\rightarrow$notify to some pointer of a shell-code.

\end{qparts}
\newpage
\section*{Problem 3}

\begin{qparts}
\item
$s1\ != NULL\ \&\&\ s2\ != NULL\ \&\&\ n1 <= size(s1)\ \&\&\ n_2\ <= size(s2)$

\item
$s1\ != NULL\ \&\&\ s\ != NULL\ \&\&\ i < size(s1)\ \&\&\ i\ < size(s)\ \&\&\ i>=0$

\item
$s2\ != NULL\ \&\&\ s\ != NULL\ \&\&\ j < size(s2)\ \&\&\ i+j\ < size(s)\ \&\&\ i+j>=0$

\item
$s\ != NULL\ \&\&\ i+j\ < size(s)\ \&\&\ i+j>=0$
\end{qparts}


\newpage
\section*{Problem 4}

\begin{qparts}
\item
Since it' $unit64\_t$, we know that each $memcpy()$ will write 8 Bytes.\\
So if $i= 18$, it will past end of a.\\
\tab + $i = 0 \rightarrow a[0\cdots 7]$\\
\tab + $i = 1 \rightarrow a[1\cdots 8]$\\
\tab + $i = 2 \rightarrow a[2\cdots 9]$\\
\tab\tab $\cdots$\\
\tab + $i = 18 \rightarrow a[18\cdots 25]$

\item
\boxed{Any\ Value\ that\ is\ greater\ or\ equal\ to\ 19}



\end{qparts}

\newpage
\section*{Problem 5}

\begin{qparts}
\item
%$\frac{1}{\frac{6}{2^{64}}} = \boxed{\frac{2^{64}}{6}}$
%(2^64 - 6) *(1/7) +1
$$\boxed{(2^{64}-6)\times\Big(\frac{1}{7}\Big) + 1}$$
\item
P = $(x_0 >= 0) \land (y_0 >= 0) \land (x_0 == 8) \land (y_0 <= 19)$
\item
Q = $(y_0 >= 0) \land (y_0 <= 13)$

\item
$(P \land \neg Q) = [(x_0 >= 0) \land (y_0 >= 0) \land (x_0 == 8) \land (y_0 <= 19)] \land [(y_0 < 0) \lor (y_0 > 13)]$\\
$\boxed{(x_0 == 8) \land (14<= y_0<=19)}$
\end{qparts}


\newpage
\section*{Problem 6}

\begin{qparts}
\item
I would allow a certain (variable) number of miss-types, so that allow the users to access the system if their number of mis-types is less than or equals to the variable.

\item
If we allow a certain number of mis-types, the authorized users will have more more chances to log in. Hence, "False negative" will be decrease because because the authorized knows their passwords.\\
Also, "False positive" will be increase because we allow mis-types in the passwords, so the unauthorized users will have more chances of hacking and logging in.\\
\item

Let x be the \# of errors that we will allow\\
$$P(x\leq k)\geq 0.998$$
$$\sum\limits_{i=0}^k P(x=i)$$
$$\sum\limits_{i=0}^k \binom{10}{i}\cdot (98)^{10-i} \cdot (0.02)^2$$
If the users make 2 or less typos, let them log-in to the system\\
"False negative" rate $<$ 0.002\\
"False positive" rate $=\frac{1}{2^{30}}$ 


\end{qparts}

\newpage
\section*{Problem 7}

\begin{qparts}
\item
1:\tab$xorl\ \%ecx\ \%ecx $\\
2:\tab$addl\ \$42\ \%ecx $\\
3:\tab$movl\ \%ecx\ \%eax $\\

\item
D: 2ef4f00d\tab(xchg \%eax \%ecx)\\
C: 2ef4cafe\tab(addl \$42 \%ecx)\\
B: 2ef4face\tab(xorl \%ecx \%ecx)\\
A: Doesn't matter\\

\item
E: 0xbfdecaf0\tab(./a.out)\\
D: 0xbfdecae0\tab(gcc /tmp/foo.c)\\
C: 0x1ffa4b28\tab(System)\\
B: 0x1ffa4b28\tab(System)\\
A: doesn't matter


\item
H: 0xbfdecaf8\tab(rm a.out/tmp/foo.c)\\
G: 0xbfdecaf0\tab(./a.out)\\
F: 0x1ffa4b28\tab(System)\\
E: 0x1ffa4b28\tab(System)\\
D: 0xbfdecae0\tab(gcc /tmp/foo.c) \\
C: 0x2ef4dead\tab(pop) \\
B: 0x1ffa4b28\tab(System)\\
A: doesn't matter


\end{qparts}


\end{document}
